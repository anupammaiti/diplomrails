%
% Diplomarbeit mit LaTeX
% ===========================================================================
% This is part of the book "Diplomarbeit mit LaTeX".
% Copyright (c) 2002-2005 Tobias Erbsland, Andreas Nitsch
% See the file diplomarbeit_mit_latex.tex for copying conditions.
%

\chapter{Aufbau gro�er Dokumente}
\label{sec:aufbaugrosserdokumente}

Bei gro�en Dokumenten geht schnell der �berblick verloren. Deshalb solltest du dein Dokument in einzelne Dateien aufteilen. Dazu steht dir der Befehl \texttt{\textbackslash include} und \texttt{\textbackslash input} zur Verf�gung, mit welchen du eine weitere Datei an dieser Stelle einbinden kannst.

TeXnicCenter unterst�tzt dich bereits darin, indem er f�r dein Projekt ein eigenes Unterverzeichnis anlegt. Ich gehe in den folgenden Beispielen davon aus, dass du eine Diplomarbeit zu einem Onlineshop schreibst. Beim Erstellen von deinem TeXnicCenter Projekt (siehe dazu Kapitel \ref{sec:erstellenneuesprojekt} auf Seite \pageref{sec:erstellenneuesprojekt}) gibst du also als Projektname \enquote{onlineshop} an. TeXnicCenter erstellt dir dann in dem gew�hlen Unterverzeichnis ein Verzeichnis mit dem Namen \enquote{onlineshop} und legt darin die Datei \enquote{onlineshop.tex} an. Diese Datei definiert TeXnicCenter automatisch auch als Hauptdatei.

\section{Aufbauen einer Verzeichnisstruktur}

In diesem Unterverzeichnis legst du jetzt folgende Verzeichnisse an: bilder, listings, bibliographie und kapitel. Nat�rlich kannst du jede beliebige Bezeichnung verwenden. Ich pers�nlich bevorzuge englische Namen: images, listings, bibliography und chapters. 

Jetzt sieht dein Verzeichnisbaum folgenderma�en aus:

\begin{verbatim}
onlineshop -+- bibliographie
            |
            +- bilder
            |
            +- listings
            |
            +- kapitel
\end{verbatim}

\section{Anlegen der einzelnen Dateien}

\subsection{Die Hauptdatei}

In deiner Hauptdatei f�gst du praktisch nur \texttt{\textbackslash include} und ein \texttt{\textbackslash input} Befehl ein, welche weitere Dateien einbinden. So eine Hauptdatei siehst du in Listing \ref{lst:beispiel05}.

\lstinputlisting[caption=Hauptdatei des Projekts (onlineshop.tex),
	label=lst:beispiel05, frame=tb]%
	{listings/beispiel05.tex}

Das hat verschiedene Vorteile:

\begin{itemize}
	\item Durch einfaches Austauschen der einzelnen \texttt{\textbackslash include} Befehle kannst du die Kapitel neu anordnen.
	\item Indem du z.\,B. den Befehl \texttt{\textbackslash includeonly\{kapitel/ersteskapitel\}} als ersten Befehl in Listing \ref{lst:beispiel05} einf�gst, wird nur genau das Kapitel \enquote{Erstes Kapitel} erzeugt. Was nat�rlich sehr viel schneller geht, als wenn das ganze Dokument erstellt werden m�sste. Das spart dir viel Zeit, wenn du die Darstellung eines einzelnen Kapitels optimierst.
	\item Wenn du sp�ter �hnliche Dokumente erstellst, kannst du die separate Headerdatei kopieren und wiederverwenden.
\end{itemize}

Wie du auch siehst, kannst du die Dateiendung \enquote{.tex} beim \texttt{\textbackslash include} und beim \texttt{\textbackslash input} Befehl weglassen. \DMLLaTeX \ f�gt diese Endung automatisch an den Dateinamen an.

\subsection{Der Unterschied zwischen \texttt{\textbackslash include} und \texttt{\textbackslash input}}

Wenn du nocheinmal das Listing \ref{lst:beispiel05} betrachtest, siehst du, dass ich den Kopfbereich mit dem \texttt{\textbackslash input}-, die Kapitel aber mit dem \texttt{\textbackslash include}- Befehl eingebunden habe.

Der \texttt{\textbackslash input} Befehl f�gt die angegebene Datei genau an der angegebenen Stelle ein, der \texttt{\textbackslash include} Befehl macht jedoch noch mehr. Bevor die Datei an der angegebenen Stelle eingebunden wird, wird noch ein \texttt{\textbackslash clearpage} Befehl eingef�gt. Dieser Befehl schreibt ausstehende Dinge wie Fu�noten und Floats (siehe Kapitel \ref{sec:tabellenundbilder}) noch fertig und beginnt eine neue Seite. Am Ende der eingef�gten Datei merkt sich \DMLLaTeX \ alle Z�hlerst�nde usw. und speichert sie als \enquote{.aux} Datei ab.

F�gst du jetzt den Befehl \texttt{\textbackslash includeonly} am Anfang des Hauptdokuments ein, damit nur ein einzelnes Kapitel erzeugt wird, dann rekonstruiert \DMLLaTeX \ aus diesen \enquote{.aux} Dateien die Z�hlerst�nde und nummeriert das einzelne Kapitel genauso, als w�rde es mitten in deinem Dokument stehen.

\subsection{Der Header}

Jetzt erstellst du eine neue Datei und speicherst sie in deinem Projektverzeichnis unter dem Dateinamen \enquote{header.tex}. In dieser Datei baust du den ganzen Kopfbereich deines \DMLLaTeX-Dokuments auf. So eine Headerdatei siehst du in Listing \ref{lst:beispiel06}.

\lstinputlisting[caption=Headerdatei des Projekts (header.tex),
	label=lst:beispiel06, frame=tb]%
	{listings/beispiel06.tex}

\subsection{Die Kapitel}

F�r jedes Kapitel erstellst du jetzt eine separate Datei. Diese Dateien solltest du nicht nummerieren, sondern nach ihrem Inhalt benennen, sonst erweist sich das einfache Verschieben oder Austauschen der einzelnen Kapitel als sehr verwirrend. Wenn pl�tzlich die Kapitel in der Reihenfolge 3, 2, 5, 1 und 4 in dem Hauptdokument eingebunden werden, geht schnell die �bersicht verloren.

Eine solche Kapiteldatei siehst du z.\,B. in Listing \ref{lst:beispiel07}

\lstinputlisting[
	caption=Kapiteldatei des Projekts (einfuehrung.tex im Verzeichnis \enquote{kapitel}),
	label=lst:beispiel07, frame=tb]%
	{listings/beispiel07.tex}

\subsection{Die Titelseite}

Am Schluss erstellst du noch die Titelseite in einer separaten Datei. Dabei kannst du z.\,B. die Umgebung \enquote{titlepage} verwenden.

Eine Datei, welche die Titelseite enth�lt, siehst du z.\,B. in Listing \ref{lst:beispiel08}. Du siehst auch das nach der Titelseite auch das Inhaltsverzeichnis, das Abbildungs- und Tabellenverzeichnis eingef�gt wird.

\lstinputlisting[
	caption=Titelseite des Projekts (titelseite.tex im Verzeichnis \enquote{kapitel}),
	label=lst:beispiel08, frame=tb]%
	{listings/beispiel08.tex}

\section{Weitere Aufteilungen}

\subsection{Gro�e Kapitel}

Falls die einzelnen Kapitel zu gro� werden, kannst du auch im Unterverzeichnis \enquote{kapitel} f�r jedes Kapitel weitere Unterverzeichnisse anlegen und dort die einzelnen Abschnitte als Dateien anlegen.

Dazu erstellst du eine Datei, welche alle diese Abschnitte mit einem \texttt{\textbackslash input} Befehl einbindest. F�r diese feineren Unterteilungen innerhalb eines Kapitels solltest du nicht den\\ \texttt{\textbackslash include} Befehl verwenden, weil dieser immer einen Seitenumbruch einf�gen w�rde.

\subsection{Viele Bilder}

Auch im Verzeichnis mit den Bildern kannst du mit weiteren Unterverzeichnissen die �bersicht zur�ckgewinnen. Eine Aufteilung kannst du z.\,B. nach Kapiteln oder nach Kategorien, denen man die Bilder zuordnen k�nnte, machen.

%
% EOF
%
