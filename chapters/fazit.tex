\chapter{Zusammenfassung}

\section{Fazit}

Im Rahmen dieser Arbeit konnten die ausgewählten Ruby on Rails Web Content Management Systeme Alchemy CMS, Browser CMS, Locomotive CMS und Refinery CMS auf ihre Webpublishing-Fähigkeiten untersucht werden. Die Systeme zeigten dabei vor allem in den Webpublishingdisziplinen Publikation und Freigabe große Schwächen. Verwertbare Workflow-Funktionalitäten konnte dabei vor alle

Die in Kapitel \ref{chap:probleme} aufgezeigten Implementierungsdetails zeigen darüber hinaus, das die Systeme hinsichtlich ihres Datenbankdesigns und der User-Interface-Umsetzung Optimierungspotenzial besitzen.
Ein Einsatz der Systeme kann daher nur empfohlen werden, wenn in einer Voranalyse die Anforderungen an die tatsächlich umzusetzende Internetseite genau definiert und abgeschätzt werden. Eine nachträgliche Erweiterung der bestehenden Systeme ist möglich, erfordert jedoch einen erhöhten Mehraufwand im Vergleich zu verbreiteten Lösungen wie Typo3 oder Drupal, die im Bereich des Webpublishing bereits viele der hier untersuchten Funktionalitäten erfüllen.



\section{Ausblick}
Die vorgestellten WCMS sind zum Teil erst innerhalb des letzten Jahres (2010) an die Öffentlichkeit übergeben wurde.
