\chapter{Zusammenfassung}

\section{Fazit}

Im Rahmen dieser Arbeit konnten die ausgewählten Ruby on Rails Web Content Management Systeme Alchemy CMS, Browser CMS, Locomotive CMS und Refinery CMS auf ihre Webpublishing-Fähigkeiten untersucht werden. Dabei zeigte sich, dass der allgemein in Content Management Systemen etablierte Redaktionsprozess (Erstellung, Verwaltung, Publikation und Archivierung) sehr unterschiedlich unterstützt wird.
Ein Einsatz der Systeme kann daher nur empfohlen werden, wenn in einer Voranalyse die Anforderungen an die tatsächlich umzusetzende Internetseite genau definiert und mit dem gewünschten Rails WCMS abgeglichen werden. Zwar bietet Rails komfortable Möglichkeiten zur Anpassung der Systeme, dies erfordert jedoch einen erhöhten Mehraufwand im Vergleich zu verbreiteten Lösungen wie Typo3 oder Drupal, die im Bereich des Webpublishing bereits einen Großteil der hier untersuchten Funktionalitäten erfüllen.
Die in Kapitel \ref{chap:probleme} aufgezeigten Implementierungsdetails zeigen darüber hinaus, das die Systeme hinsichtlich ihres Datenbankdesigns und der User-Interface-Umsetzung Optimierungspotenzial besitzen. Vor allem der in Rails übliche Einsatz von Generatorskripten und HTML-Views bei der Erweiterungsentwicklung lässt die Arbeit mit diesen Systemen schnell unübersichtlich und aufwendig erscheinen.

\section{Ausblick}
Die vorgestellten WCMS sind zum Teil erst innerhalb der letzten 2 Jahre (2009) an die Öffentlichkeit übergeben wurden. Durch die Leistungen der Open Source-Bewegung ist daher mit weiteren funktionalen und technischen Verbesserungen zu rechnen. Die Umsetzung eines rails-basierten Content Repository kann dabei einen wichtigen Entwicklungsimpuls innerhalb der Ruby on Rails Web Content Management Systeme bedeuten.

