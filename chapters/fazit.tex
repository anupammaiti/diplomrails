\chapter{Zusammenfassung}

\section{Fazit}

Im Rahmen dieser Arbeit konnten die ausgewählten Ruby on Rails Web Content Management Systeme Alchemy CMS, Browser CMS, Locomotive CMS und Refinery CMS auf ihre Webpublishing-Fähigkeiten untersucht werden. Dabei zeigte sich, dass der allgemein in Content Management Systemen etablierte Redaktionsprozess (Erstellung, Verwaltung, Publikation und Archivierung) sehr unterschiedlich unterstützt wird.
Ein Einsatz der Systeme kann daher nur empfohlen werden, wenn in einer Voranalyse die Anforderungen an die tatsächlich umzusetzende Internetseite genau definiert und mit den gebotenen Funktionalitäten der Rails WCMS verglichen wird. Trotz der in Rails vorhandenen Möglichkeit zur individuellen Anpassung der Systeme ist der damit erforderliche Mehraufwand nicht vertretbar. Vorallem der Vergleich zu verbreiteten Lösungen wie Typo3 oder Drupal, die im Bereich des Webpublishing bereits einen Großteil der hier untersuchten Funktionalitäten erfüllen, lässt eine eigenständige Weiterentwicklung der bestehenden Rails WCMS nur beschränkt sinnvoll erscheinen.
Die in Kapitel \ref{chap:probleme} aufgezeigten Implementierungsdetails zeigen darüber hinaus, das die Systeme hinsichtlich ihres Datenbankdesigns und der User-Interface-Umsetzung Optimierungspotenzial besitzen. Vor allem der in Rails übliche Einsatz von Generatorskripten und HTML-Views bei der Erweiterungsentwicklung lässt die Arbeit mit diesen Systemen schnell unübersichtlich und aufwendig erscheinen. Zusätzlich ergeben sich starke Abhängigkeiten zwischen dem WCMS und den vorhandenen Erweiterungen.

\section{Ausblick}
Die vorgestellten WCMS sind zum Teil erst innerhalb der letzten 2 Jahre (2009) an die Öffentlichkeit übergeben wurden. Durch die Leistungen der Open Source-Bewegung ist daher mit weiteren funktionalen und technischen Verbesserungen zu rechnen. Die Umsetzung eines rails-basierten Content Repository kann dabei einen wichtigen Entwicklungsimpuls innerhalb der Ruby on Rails Web Content Management Systeme signalisieren und eine sinnvolle Vereinfachung für Erweiterungsentwicklungen bedeuten.
Die Implementierungsdetails der ausgewählten Systeme bieten tendenziell viel Freiraum für neue Ideen sowie technische und strukturelle Verbesserungen.

