
% Diplomarbeit mit LaTeX
% ===========================================================================
% This is part of the book "Diplomarbeit mit LaTeX".
% Copyright (c) 2002-2005, 2007 Tobias Erbsland
% Copyright (c) 2005 Andreas Nitsch
% See the file diplomarbeit_mit_latex.tex for copying conditions.
%

\chapter{Literaturempfehlungen}
\index{Literatur}\index{Bucher@B�cher}

\section{B�cher und Internetadressen}

Setzt du konsequent die Hilfsmittel ein, die \DMLLaTeX \ dir bietet, so wirst du auf jeden Fall ein optisch ansprechendes Dokument erhalten. Wenn du dabei auch noch die Regeln beachtest und dich um Warnungen und Fehlermeldungen k�mmerst, dann erh�lst du ein optisch �berzeugendes Dokument.

Ob deine Diplomarbeit f�r den Leser besonders spannend ist, h�ngt sowohl vom jeweiligen Thema und von den Interessen des Lesers ab. Trotzdem kannst du noch einiges optimieren um deine Arbeit f�r den Leser\footnote{Besonders f�r diejenigen, die dir eine Note daf�r geben werden} spannender und sicher auch unterhaltsamer zu gestalten.

Mit \enquote{einfachen W�rtern} und \enquote{sch�nen S�tzen} kann der Leser dein Werk in vollen Z�gen genie�en.

Falls aber Schreiben nicht zu deinem St�rken z�hlt, k�nnten dir die folgenden Literaturtipps eine gute Hilfe sein. Einige davon kannst du kostenlos �ber das Internet beziehen.

\begin{itemize}
	\item{Claudia Fritsch: ''Schreiben f�r die Leser''\cite{fritsch:schreiben_fuer_die_Leser}}
	\item{Wolf Schneider: ''Deutsch f�rs Leben''\cite{schneider:deutsch_fuers_leben}}
\end{itemize}

Weitere und ausf�hrlichere \footnote{...daf�r aber nicht so speziell auf das Erstellen einer Diplomarbeit ausgelegte} Literatur zu \DMLLaTeX \ findest du  kostenlos im Internet:

\begin{itemize}
	\item{Manuela J�rgens: \enquote{\DMLLaTeX - eine Einf�hrung und ein bisschen mehr}\cite{juergens:latex1}}
	\item{Manuela J�rgens: \enquote{\DMLLaTeX - fortgeschrittene Anwendungen}\cite{juergens:latex2}}
\end{itemize}

Zudem gibt es viele gute B�cher welche weitere Details von \DMLLaTeX \ beschreiben. Folgende B�cher k�nnen als Standardwerke in Sachen \LaTeX \ bezeichnet werden:

\begin{itemize}
	\item{Frank Mittelback, Michael Goossens: \enquote{Der \DMLLaTeX \ Begleiter}\cite{DerLaTeXBegleiter}}
	\item{Helmut Kopka: \enquote{\DMLLaTeX: Band 1 - Eine Einf�hrung}\cite{kopka:band1}}
	\item{Helmut Kopka: \enquote{\DMLLaTeX: Band 2 - Erg�nzungen}\cite{kopka:band2}}
	\item{Helmut Kopka: \enquote{\DMLLaTeX: Band 3 - Erweiterungen}\cite{kopka:band3}}
	\item{Leslie Lamport: \enquote{\DMLLaTeX, A Document Preparation System, User's Guide and Reference Manual}\cite{lamport:latex}}
\end{itemize}




