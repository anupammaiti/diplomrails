%
% Diplomarbeit mit LaTeX
% ===========================================================================
% This is part of the book "Diplomarbeit mit LaTeX".
% Copyright (c) 2002-2005 Tobias Erbsland, Andreas Nitsch
% See the file diplomarbeit_mit_latex.tex for copying conditions.
%

\chapter{Tastenkombinationen im TeXnicCenter}

Hier noch die wichtigsten Tastenkombinationen f�r eine schnelle Bedienung vom TeXnicCenter. Die Tastenkommandos lassen sich nat�rlich anpassen. �ber das Hilfemen� \enquote{?} $\Rightarrow$ \enquote{Tastaturbelegung...} l�sst sich die Tastaturbelegung auch anzeigen oder ausdrucken. 

\begin{table}[H]
	\begin{center}
	\begin{tabular}{rl}
		\textbf{Tastenkombination} & \textbf{Beschreibung} \\ \hline
		
		Ctrl + 0-9 & Setzen von nummeriertem Lesezeichen \\
		Alt + 0-9 & Zu nummerierten Lesezeichen springen \\
		
		Ctrl + A & Markiert das ganze Dokument \\
		Ctrl + Alt + A & Einf�gen einer Abbildungsumgebung \\
		
		Ctrl + B & Gehe zu letzter �nderung \\
		Ctrl + Alt + B & Einf�gen einer Beschreibungsliste \\
		
		
		Ctrl + C & Markierten Text in die Zwischenablage kopieren \\
		
		
		Ctrl + E & Hervorheben des markierten Textes \\
		Ctrl + Alt + E & Einf�gen eines Aufz�hlungseintrags \\
		
		Ctrl + F & Suchen eines Textes \\
		Ctrl + Alt + F & Einf�gen einer Fu�note \\
		
		Ctrl + Alt + G & Einf�gen einer Grafik (Dialog) \\
		
		Ctrl + H & Ersetzen eines Textes \\
		
		Ctrl + K & Kursivsetzen des markierten Textes \\
	
		Ctrl + Alt + L & Einf�gen eines Beschreibungsbeitrags \\
	
		Ctrl + Alt + N & Einf�gen einer Nummerierung \\
		
	
		Ctrl + S & Speichen des aktuellen Dokuments \\
		Ctrl + Shift + S & Speichern aller Dokumente \\
		Ctrl + Alt + S & Erzeugen eines neuen Titels \\
		
		Ctrl + Alt + T & Einf�gen einer Tabelle (Dialog) \\
		
		Ctrl + V & Text aus Zwischenablage einf�gen \\
		
		Ctrl + X & Markierten Text ausschneiden und in die Zwischenablage \\
		
		Ctrl + Z & Undo, macht die letzte Aktion R�ckg�ngig \\
		Ctrl + Alt + Z & Einf�gen einer Aufz�hlung \\	
		
		F3 & Nach suchen, zum n�chsten Treffer springen \\
		
		F5 & Ausgabedatei Betrachten \\
		F7 & Projekt Kompilieren \\
		Ctrl + F7 & Aktuelle Datei kompilieren \\
		
		F9 & N�chster Fehler anzeigen \\
		Shift + F9 & Vorheriger Fehler anzeigen \\
			
		F10 & N�chste Warnung anzeigen \\
		Shift + F10 & Vorherige Warnung anzeigen \\

		F11 & N�chste volle/leere Box anzeigen \\
		Shift + F11 & Vorherige volle/leere Box anzeigen \\
	\end{tabular}
	\end{center}
	\caption{Tastenkombinationen im TeXnicCenter}
\end{table}
