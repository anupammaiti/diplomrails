%
% Diplomarbeit mit LaTeX
% ===========================================================================
% This is part of the book "Diplomarbeit mit LaTeX".
% Copyright (c) 2002-2005 Tobias Erbsland, Andreas Nitsch
% See the file diplomarbeit_mit_latex.tex for copying conditions.
%

\chapter{Einleitung}

\section{Ausgangslage}
%\label{sec:Ausgangslage}



%\begin{quote}
%\enquote{Es gibt Alternativen zu WYSIWYG\footnote{What You See Is What You Get} Textverarbeitungen}.
%\end{quote}

Die Skriptsprache PHP gehört weltweit zu den meist genutzten serverseitigen Programmiersprachen. Im August 2011 sind über 75 Prozent der dynamisch generierten Internetseiten mit dem PHP Hypertext Preprocessor erzeugt wurden\footnote{\href{http://w3techs.com/}{W3tech} erstellt täglich eine aktualisierte Auflistung über die Verwendung von serverseitigen Programmiersprachen. Es werden dabei die nach dem Alexia Ranking eine Million beliebtesten Internetseiten auf ihre Konfiguration untersucht.}.

\begin{figure}[h]
\begin{center}
\label{fig.programmingusage}
\includegraphics[scale=0.65]{images/Einleitung/serverseitigeScriptsprachen.png}
\caption{Nutzung verschiedener Programmiersprachen auf Servern}
\end{center}
\end{figure}

Auch im Bereich der Web Content Management Systeme\footnote{Im folgenden wird für den Begriff Web Content Management Systeme die Abkürzung WCMS verwendet} spiegelt sich diese Dominanz wider. Betrachtet man die Angaben des Content Management Portals cmsmatrix.org\footnote{\href{http://cmsmatrix.org}{http://cmsmatrix.org} ermöglicht eine Gegenüberstellung der Funktionalitäten von Content Management Systemen unterschiedlicher Prgrammiersprachen.}, existieren neben den vor allem in Deutschland verwendeten Open Source-Lösungen Typo3, Drupal, Contao oder Joomla! über 500 weitere in PHP implementierte Web Content Management Systeme unterschiedlichster Ausprägung und Qualität.
Ruby als Programmiersprache findet hingegen nur bei etwa 1 Prozent der erfassten Server Verwendung. Die dabei umgesetzten Projekte sind meist individuelle, browser-basierte Applikationen, die für Unternehmen und deren spezifisches Geschäftsfeld entwickelt wurden. Bekannte Vertreter sind hier u.a. die webbasierte Projektmanagement-Applikation Basecamp von 37signals\footnote{Projektseite von Basecamp:\href{http://basecamphq.com}{ http://basecamphq.com/}}, der Microblogging-Dienst Twitter\footnote{
Großteile der Programmierung von Twitter basierten bis April 2011 auf dem Ruby on Rails Framework.} und der webbasierte Hosting-Dienst Github\footnote{Github greift neben Ruby on Rails noch  Webframeworks und Technologien zurück.
} für Software-Entwicklungsprojekte.
Diese individuellen Lösungen werden dabei meist unter Zuhilfenahme  eines Web Applications Frameworks realisiert, das den Entwicklungsprozess unterstützt und vereinfacht.


%\begin{}
%\includegraphics[width=\linewidth]{images/Einleitung/serverseitigeScriptsprachen.png}
%\caption{}
%\label{fig:beispiel}
%\end{figure}

\section{Motivation und Zielsetzung}

Das Webframework Ruby on Rails\footnote{Im weiteren Verlauf dieser Arbeit wird für das Webframework Ruby on Rails die Kurzform Rails verwendet.} hat sich seit der Veröffentlichung der Version 1.0 im Juli 2004 zu einem der bekanntesten Webframeworks der Ruby Fangemeinde entwickelt.
Startups\footnote{Der Ebgriff Startup bezeichnet hier junge Unternehmen, die sich mit ihrem neuartigen, meist innovativen Produkt noch nicht am Markt etabliert haben.} sowie etablierte Unternehmen greifen zunehmend auf das Rails Framework zurück, um ihre webbasierten Geschäftsideen und -modelle zu realisieren.
Wird neben der Webapplikation zusätzlich eine Internetseite zur Repräsentierung der Unternehmung benötigt, haben sich in der Praxis folgende zwei Lösungsansätze herausgebildet:

\begin{enumerate}
\item{
Bei geringem Umfang der zusätzlichen Internetseite werden die Inhalte manuell in HTML-Dateien angelegt und anschließend in die Rails-Anwendung integriert. Komfortable Möglichkeiten der Content-Verwaltung werden nicht angeboten oder später rudimentär nach implementiert. Änderungen der Inhalte sind teilweise mit erhöhtem Aufwand verbunden oder erfordern zusätzliche Programmierkenntnisse\footnote{Änderungen am Quellcode von Rails-Anwendungen erfordern immer einen Neustart der Anwendung.}.
}

\item{
Komplexe Internetseiten mit vielen Inhalten werden über ein Web Content Management System eines Drittanbieters realisiert. Die Rails-Anwendung fungiert als Zwischenstation und leitet bestimmte Anfragen an das externe WCMS weiter.
}
\end{enumerate}

Während der erste Lösungsansatz bei wenigen Inhalten noch vertretbar ist, erfordert die Verwendung eines externen WCMS zusätzlichen Installations- und Wartungsaufwand. Weiterhin erhöht sich der Bedarf an Programmierern, da neben Ruby nun auch andere Programmiersprachen Verwendung finden können.

Ziel der vorliegende Arbeit ist es daher, die Möglichkeiten einer rails-basierten Web Content Management Verwaltung zu untersuchen, um so den Einsatz eines externen WCMS überflüssig werden zu lassen.



\section{Aufbau der Arbeit}
Die vorliegende Diplomarbeit gliedert sich in sechs wesentliche Abschnitte.

Im ersten Abschnitt, in Kapitel 2, werden die für diese Arbeit notwendigen theoretische Grundlagen zu Web Content Management Systemen und dem Web Framework Ruby on Rails geschaffen. Es wird ebenfalls auf die neuen Funktionalitäten der Version 3.1 des Frameworks eingegangen. Zusätzlich werden die grundlegenden Charakteristika der Programmiersprache Ruby erläutert.
Im zweiten Abschnitt, in Kapitel 3, folgt eine Analyse der bestehenden Ruby on Rails 3 Web Content Management Systeme Alchemy, RefineryCMS, BrowserCMS und Lokomotive CMS. Die ausgewählten Systeme werden dabei vorgestellt und anschließend mit Hilfe eines externen Kriterienkatalogs auf ihre Leistungsfähigkeit untersucht.
Kapitel 4 überprüft die analysierten WCM-Systeme auf eventuell vorhandene konzeptionelle und programmiertechnische Schwachstellen.
Darauf aufbauend werden in Kapitel 5 mögliche Lösungsansätze demonstriert und dafür notwendige theoretische Grundlagen herausgearbeitet.
Kapitel 6 schließt die Arbeit mit einem Ausblick auf weitere Entwicklungen sowie einem Fazit ab.
%
% EOF
%

